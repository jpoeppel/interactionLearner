\chapter{State of the art \label{chap:stateOfTheArt}}


%What kinds of interaction models are there
%Incremental/online models
%Memory based approaches
%What have others tried to solve this/similar problems?

Because of the importance for robotics, a lot of research is being done regarding object interactions. 
Nishide et al. trained a \gls{rnn} in order to predict object dynamics directly from the visual input of a camera \cite{nishide2008predicting}.
In order to solve this more difficult problem they train their system roughly 20 minutes on prerecorded data.
In \cite{levine2015learning} Levine et al. achieved great results in learning  policies of complex dynamics %TODO \gls{policy}.
such as assembling plastic toys. They require much fewer training iterations than earlier policy search approaches, i.e in \cite{otherPolicySearchs}.
However, the main difference to the given problem for this thesis is that Levine et al. fix the manipulated object to the actuator.




\begin{itemize}
	\item \cite{pushing} Paper using similar idea of instance based memory in order to predict pushing
	\begin{itemize}
		\item restricted to only pushing interactions (no getting closer etc learned by model)
		\item is mainly used to move object to target position
		\item General strategy is the same as in my model: First figure out preconditions, then perform best action
		\item Uses nearest neighbor to find best action
		\item Works for complex shaped objects
		\item Paper only shows results based on position, not orientation
		
	\end{itemize}
	\item \cite{nishide2008predicting} RNN and HNN system that learns interactions between objects from camera images
	\begin{itemize}
		\item Tries to solve more complex problem (includes feature extraction from images in the system)
		\item But is learned in a batch mode. Prerecorded by controlling the robot and training on recorded images (~ 20min offline)
		\item They call it \textit{active sensing} regardless
	\end{itemize}
	\item \cite{baranes2013active} Robot architecture for active, online learning of forward and inverse models
	\begin{itemize}
		\item Limited to only the robot, not including external objects and their interactions
		\item Highlights some of the challenges when trying to (actively) learn online
	\end{itemize}
	\item \cite{flanagan2006control} Recent (2006) paper about human object manipulation control strategies
	\item \cite{moldovan2012learning} Learning object affordances with probabilistic model. Actually extends to multiple objects
	\item \cite{levine2015learning} Impressive (batch) results of object manipulation.
	\begin{itemize}
		\item Restriction is that objects are fixed to robot
		\item Basically \textit{only} learns a trajectory
	\end{itemize}
	\item \cite{contactPrediction} Exemplar-based learning of contact/support interaction between (multiple) objects.
	\begin{itemize}
		\item Contact distributions are estimated from contact points extracted from the point clouds of objects
		\item Similarity between distributions determines whether the given contact allows a special interaction
		\item Binary classification if a certain interaction is possible/happening
		\item Does not make predictions about the future state of the interacted objects
		\item Has not been trained online, but rather with previously labeled training data
	\end{itemize}
	\item TODO
	\begin{itemize}
		\item Look at experience graphs
		\item Can be mentioned here as well (\cite{phillips2012graphs}) as somewhat similar idea regarding the use of previous experiences, maybe better in introduction?
	\end{itemize}
\end{itemize}
