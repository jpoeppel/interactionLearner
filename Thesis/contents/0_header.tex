% XeTeX-Vorlage für Abschlußarbeiten an der Universität Bielefeld
% ---------------------------------------------------------------------
% Stand, 07. Juni 2010
%
% Timo Reuter (treuter@cit-ec.uni-bielefeld.de), AG Semantic Computing
% Exzellenzcluster für Cognitive Interaction Technology
%
% Dies Vorlage setzt die Schriftart FagoNo sowie einen XeTeX-Kompiler 
% voraus. Alle Dokumente müssen zwingend mit Unicode (UTF-8) als
% Kodierung abgespeichert werden.

\documentclass[
  a4paper,              % Wir verwenden A4 Papier
  %oneside,             % Einseitiger Druck
  twoside,              % Zweiseitiger Druck
  12pt,                 % Große Schrift, besser geeignet für A4
  halfparskip,          % Halbe Zeile Abstand zwischen Absätzen
  chapterprefix,        % Kapitel mit 'Kapitel' anschreiben
  headsepline,          % Linie nach Kopfzeile
  footsepline,          % Linie vor Fußzeile
  bibtotocnumbered,     % Literaturverzeichnis im Inhaltsverzeichnis numeriert einfügen
  DIV=15,
  BCOR=8mm
]{scrbook}

% Kopf-/ und Fußzeilen
\usepackage{footmisc}

% Parts haben im Inhaltsverzeichnis keine Seitenzahlen
\makeatletter
\let\partbackup\l@part
\renewcommand*\l@part[2]{\partbackup{#1}{}}
\makeatother

% Zeichencodierung des Dokuments und des Zeichensatzes.
\usepackage{pifont}
\usepackage{fontspec}
\usepackage{xunicode}
\usepackage{xltxtra}
\usepackage{textcomp}
\usepackage{csquotes}

% Lokalisierung auf deutsch (Silbentrennung usw.)
%\usepackage[german]{babel}
\usepackage[english]{babel}

% Paket um Grafiken im Dokument einbetten zu können.
\usepackage{graphicx}
\graphicspath{{images/}}

% PDF-Merkmale
\usepackage[
  pdftitle={Memory-based exploratory online learning of simple object manipulation},            % Titel des PDF Dokuments.
  pdfauthor={Jan Pöppel},                 % Autor des PDF Dokuments.
  pdfsubject={Memory-based exploratory online learning of simple object manipulation},          % Thema des PDF Dokuments.
  pdfcreator={Texlive, XeTeX},           % Erzeuger des PDF Dokuments.
  pdfkeywords={Keyword1, keyword2},      % Schlüsselwörter für das PDF.
  pdfpagemode=UseOutlines,               % Inhaltsverzeichnis anzeigen beim Öffnen
  bookmarksopen=false,                   % Inhaltsverzeichnisbaum geöffnet anzeigen
  pdfdisplaydoctitle=true,               % Dokumenttitel statt Dateiname anzeigen.
  pdflang=en                             % Sprache des Dokuments.
]{hyperref}

% Schriften
\defaultfontfeatures{Mapping=tex-text}
\setmainfont[ItalicFont={FagoNo}, BoldFont={FagoNoLf-Bold}]{Latin Modern Roman} %{GoudyOlSt BT}
\setsansfont{FagoNo}
%{\setmonofont{Arial}\fontsize{48pt}{\baselineskip}\selectfont}
%\usepackage{default}

% Farben definieren
\usepackage{color, colortbl}
\definecolor{LinkColor}{rgb}{0,0,0}
\definecolor{ListingBackground}{rgb}{0.85,0.85,0.85}
\definecolor{LightGrey}{rgb}{0.93,0.93,0.93}
\definecolor{LightGrey2}{rgb}{0.9,0.9,0.9}
\definecolor{DarkGrey}{rgb}{0.65,0.65,0.65}
\definecolor{UniBiGreen}{rgb}{0.0,0.457,0.336}
\definecolor{CitecOrange}{rgb}{0.93,0.453,0.004}

% Farbeinstellungen für die Verweise im PDF-Dokument.
\hypersetup{
  colorlinks=true,       % Aktivieren von farbigen Verweisen im Dokument (ohne Rahmen)
  linkcolor=LinkColor,   % Farbe festlegen.
  citecolor=LinkColor,   % Farbe festlegen.
  filecolor=LinkColor,   % Farbe festlegen.
  menucolor=LinkColor,   % Farbe festlegen.
  urlcolor=LinkColor,    % Farbe von URL's im Dokument.
  bookmarksnumbered=true % Überschriftsnummerierung im PDF Inhalt anzeigen.
}

% Beschriftungen für Bilder und Tabellen
\setcapindent{1em} % Zeilenumbruch bei Bildbeschreibungen.
\setkomafont{captionlabel}{\fontspec[Color=656565, Scale=0.7]{FagoNoLf-Bold}} % Beschriftungen für Bilder und Tabellen
\setkomafont{caption}{\fontspec[Color=656565, Scale=0.7]{FagoNoLf-Bold}} % Beschriftungen für Bilder und Tabellen
\setkomafont{descriptionlabel}{\fontspec{FagoNoLf-Bold}} % Optionales Label in der description-Umgebung


% Stil der Kopf- und Fußzeilen
\usepackage[footsepline,plainfootsepline]{scrpage2}
\pagestyle{scrheadings}
\setheadsepline{}[\color{DarkGrey}]    % Kopfzeilenlinie
\setfootsepline{}[\color{DarkGrey}]    % Fußzeilenlinie
\renewcommand*{\partpagestyle}{empty}  % Parts haben keine Kopf-/Fußzeile

% Überschriften
\setkomafont{sectioning}{\normalfont\bfseries} % Titel mit Normalschrift
\addtokomafont{sectioning}{\fontspec{FagoNoLf-Bold}}
\addtokomafont{pagehead}{\fontspec[Color=656565]{FagoNoLf-Bold}}   % Seitentitel
\addtokomafont{pagenumber}{\fontspec[Color=656565]{FagoNoLf-Bold}} % Fußnoten

% Stil der Überschriften
\usepackage[Lenny]{sty/fncychapleo}

% Paket um Quelltext sauber zu formatieren.
%\usepackage[savemem]{listings}
%% Einstellungen für das 'listings' Paket.
%\lstloadlanguages{C}            % Sprache laden, notwendig wegen Option 'savemem'
%\lstset{
%	language=C,                 % Sprache des Quelltexts ist C
%	numbers=left,               % Zelennummern links
%	stepnumber=1,               % Jede Zeile numerieren.
%	numbersep=5pt,              % 5pt Abstand zum Quellcode
%	numberstyle=\tiny,          % Zeichengröße 'tiny' für die Nummern.
%	breaklines=true,            % Zeilen umbrechen, wenn notwendig.
%	breakautoindent=true,       % Nach dem Zeilenumbruch Zeile einrücken.
%	postbreak=\space,           % Bei Leerzeichen umbrechen.
%	tabsize=2,                  % Tabulatorgröße (Indent) ist 2
%	basicstyle=\ttfamily\scriptsize, % Schrift
%	showspaces=false,           % Leerzeichen nicht anzeigen.
%	showstringspaces=false,     % Leerzeichen auch in Zeichenketten ('') nicht anzeigen.
%	extendedchars=true,         % Alle Zeichen vom Latin1 Zeichensatz anzeigen.
%	backgroundcolor=\color{ListingBackground}, % Hintergrundfarbe des Quelltexts setzen.
%	captionpos=b,               % Caption unten
%    %linewidth=0.9\textwidth,   % base line width
%    xleftmargin=0.02\textwidth, % margin left
%    xrightmargin=0.02\textwidth % margin right
%}


\usepackage[linesnumbered]{algorithm2e}
\SetAlFnt{\ttfamily\small}
%\usepackage[chapter]{algorithm}
%\usepackage[noend]{algpseudocode}
%\makeatletter
%\algrenewcommand\ALG@beginalgorithmic{\ttfamily\small}
%\makeatother

%\newcommand*\Let[2]{\State #1 $\gets$ #2}
%\algrenewcommand\alglinenumber[1]{
%    {\sf\footnotesize\addfontfeatures{Colour=888888,Numbers=Monospaced}#1}}
%\algrenewcommand\algorithmicrequire{\textbf{Input:}}
%\algrenewcommand\algorithmicensure{\textbf{Output:}}
%\algnewcommand\algorithmicdata{\textbf{Data:}}
%\algnewcommand\Data{\item[\algorithmicdata]}

\usepackage{amsmath}
%\DeclareMathOperator*{\argmin}{arg\,min}
\newcommand{\argmin}{\operatornamewithlimits{argmin}}

\usepackage[e]{esvect}
\makeatletter
\def\traitfill@#1#2#3#4{%
  $\m@th\mkern2mu\relax#4#1\mkern-6.0mu%
   \cleaders\hbox{$#4\mkern0mu#2\mkern0mu$}\hfill%
   \mkern-1.5mu#3$%
}
 \makeatother
\renewcommand{\vec}[1]{\vv{#1}}

% Tabellen
\usepackage{array}                  % Paket für erweiterte Tabelleneigenschaften
\usepackage{multirow}               % Zusammengefaßte Reihen in Tabellen ermöglichen
\setlength{\arrayrulewidth}{0.7pt}
\setlength{\extrarowheight}{3pt}
\arrayrulecolor{DarkGrey}           
\usepackage{longtable}              % Ermöglicht sich über mehrere Seiten erstreckende Tabellen
\usepackage{tabularx}               % Mächtigere Tabellenumgebung


% Stil für Designboxen
\newcommand{\designbox}[2]{
  \hspace*{17pt}
  \begin{tabular*}{402pt}{p{378pt}p{0pt}}
    \begin{flushright}\fontspec[Color=656565, Scale=0.9]{FagoNoLf-Bold}#1\hspace*{-18.5pt}\vspace*{-16pt}\end{flushright}\\
    \hline
    \cellcolor{LightGrey}\fontspec{FagoNo}#2 & \cellcolor{DarkGrey}\\
    \begin{picture}(0,0)
      \put(245,13){\color{DarkGrey}\rule{151pt}{0.5pt}}
    \end{picture}
  \end{tabular*}
}

% Scientific units
\usepackage{siunitx}
\renewcommand{\m}[1]{\SI{#1}{\meter}}
\newcommand{\rad}[1]{\SI{#1}{\radian}}

%\newcommand{\met}{\si{\meter}}
%\newcommand{\rad}{\si{\radian}}

% Bibliographie-Stil
%\usepackage{apacite}         % Psychologie
%\usepackage[numbers]{natbib}  % Linguistik und Naturwissenschaften

% Silbentrennung für Wörter, die nicht vom Babel-Paket erkannt werden
\hyphenation{Dezi-mal-trenn-zeichen In-stal-la-ti-ons-as-sis-tent}

%Abkürzungen
%\usepackage{acronym}

%Glossary and Acronyms

\usepackage[nomain,toc,acronyms]{glossaries}
%\chapter{Acronyms}
%\begin{acronym}[Bash]
% \acro{ITM}{Instantaneous Topological Map}
% \acro{GNG}{Growing Neural Gas}
% \acro{LLM}{Local Linear Map}
%\end{acronym}

\newacronym{itm}{ITM}{Instantaneous Topological Map}
\newacronym{aitm}{AITM}{Adapted Instantaneous Topological Map}
\newacronym{gng}{GNG}{Growing Neural Gas}
\newacronym{llm}{LLM}{Local Linear Map}
\newacronym{nn}{NN}{Nearest Neighbour}
\newacronym{knn}{\textit{k}-NN}{k-Nearest Neighbour}
\newacronym{rnn}{RNN}{recurrent neural network}
\newacronym{bn}{BN}{Bayesian Network}
\makeglossaries
% Überschriften anpassen
%\usepackage{titlesec}
%\titleformat{\subsection}{\bfseries\large\sffamily\itshape}{}{0em}{\thesubsection~}
% Folgende Dinge tut man in TeX NICHT, da das Ganze gegen die Grundregeln des
% Textsatzes verstößt. Wer meint, es trotzdem verwenden zu müssen, um ein "Word"-ähnlicheres 
% Satzbild zu bekommen, sollte mit dem Verantwortlichen über Sinn oder Unsinn dieses 
% Vorhabens diskutieren und diese "Hacks" möglichst unterlassen.

% Zeilenabstand ändern (Standard bei TeX ist 1,2)
%\usepackage{setspace}
%\onehalfspacing % Ergibt 1,5 Zeilenabstand
%\doublespacing % Ergibt 2,0 Zeilenabstand

% Seitenränder ändern
%\usepackage{anysize}
%\marginsize{2.5cm}{2cm}{2.5cm}{2.5cm}


