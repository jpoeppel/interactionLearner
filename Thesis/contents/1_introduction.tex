\chapter{Introduction}

%Motivate necessity
%State goals
%Structure of thesis
%What is the problem that we try to solve?
%Why is it relevant to solve it?


%TODO delete? Most likely not good...
%The defining quality that any agent, artificial or natural, requires is learning. Also used for many different 
%phenomenons, the term learning at its core means adaptation. The agent or some part of it adapts in order to
%better deal with its environment. Natural agents, especially humans, display impressive learning capabilities,
%allowing them to quickly adapt to ever changing environments and new tasks.
%
%In recent years, machine learning has come a long way in understanding and reproducing some of the learning
%capabilities of humans. We now have a wide range of different techniques and algorithms that can yield good
%classification and regression results. %TODO make glossary to explain what classification and regression is
%In introduction to these can for example be found in book by Bishop \cite{bishop}.
%
%The quality of the results achieved by machine learning is however more dependent on the chosen representation
%of the problem then on the actual technique used. For example it is generally not sufficient to use the raw audio signals 
%for speech recognition \cite{speechRecog}. Instead most speech recognizers use carefully selected pre-computed
%features that allow the underlying machine learner to correctly distinguish the words. 
%
%Because finding the best features is as challenging as learning a problem itself, a lot of focus as recently 
%shifted to deep learning \cite{deepLearning}. Deep learning is a technique that is inspired by the human brain and refers to
%training neural networks with many thousands of artificial neurons. Properly trained, these networks are capable of finding 
%relevant features in the presented data by themselves \cite{deepLearningFeatures}. Currently the biggest
%problem with deep learning is the required amount of training before the network can be used successfully. Although recent
%advances such as the development of Hinton's dropout technique \cite{dropout} allows to train bigger networks with fewer data,
%this criteria still limits the potential use cases of deep learning and neural networks in general.



[TODO General introduction here]

Robotics research has advanced far enough that researchers are now 
trying to develop multi-purpose robots. While the hardware for such robots
is constantly improving, the usability of modern robots is currently more
limited by their software. Multi-purpose machines do not have the luxury of 
being limited to restricted, well known environments. Providing suitable and
sufficient training data for all possible scenarios a multi-purpose robot
might encounter is infeasible at best. Therefore, instead of trying to train
the machine beforehand, it might be better to provide it with the means to
adapt to new situations on its own. This idea falls under the concept of 
(lifelong) online adaptation or online learning. 

Apart from the general challenges in machine learning, such as generalization 
performance, lifelong online learning introduces even more challenges, such as the 
stability-plasticity-dilemma \cite{stability-plasticity}. Furthermore, in order to be able to 
continuously upgrade itself, the model needs to work in an open loop with its environment. 
Depending on the actual purpose of the robot this might add additional constraints for the update 
and query times of the used model. %Why use memory based?
The family of memory based machine learning approaches usually 
does not suffer from the stability-plasticity-dilemma due to their inherently local nature.  
Furthermore, their local nature allows them to be easily updated incrementally. Their biggest disadvantage for lifelong learning is that their update and query time usually increase with the amount of training data. Nevertheless, in the scope of this thesis, a memory based regression and classification model has been adapted and tested for its suitability for this task.

[TODO why object interaction etc]


As a smaller test environment, this thesis concentrates on incrementally learning simple object 
interactions. Since object interactions are hard to model manually, providing a robot with the 
capabilities of learning such interactions would be quite beneficial for multi-purpose robots.
By starting with as little prior world knowledge as possible, the model needs to adapt to a mostly 
unknown scenario on its own. 

Within this environment, this thesis' focus can be summarized as follows: 

\begin{itemize}
	\item Prediction and planning with minimal amount of prior world knowledge
	\item Interactive, online learning
	\item Suitability of memory based models for lifelong learning
\end{itemize}

Chapter \ref{chap:stateOfTheArt} gives an overview about how researches are 
currently approaching these questions. The concepts this thesis contributes 
are presented in chapter \ref{chap:concept}. Chapter \ref{chap:modelReal} 
describes how these concepts were realized before chapter \ref{chap:evaluation} 
presents the evaluation of these realizations. Finally, these results are discussed in chapter 
\ref{chap:conclusion}.

